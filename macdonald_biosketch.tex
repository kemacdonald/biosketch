\documentclass[svgnames,11pt]{article}
\usepackage[margin=1in]{geometry}
\usepackage[T1]{fontenc}
\usepackage{times}
\usepackage{float}
\usepackage{hyperref}
\usepackage{etoolbox}
\usepackage[english]{babel}

\usepackage{hyperref}
\hypersetup{
    colorlinks=true,
    linkcolor=blue,
    filecolor=magenta,      
    urlcolor=cyan,
}
 

\usepackage[resetlabels]{multibib}
\newcites{New}{Other work}

\patchcmd{\thebibliography}{\section*{\refname}}{}{}{}

\title{%
        \vspace{-2\baselineskip}
            \normalsize
            {\textbf{Kyle E. MacDonald }}\\
            \vspace{0.5\baselineskip}
            \hrule
            \vspace{0.5\baselineskip}
            Post-doctoral Researcher, Department of Communications, University of California, Los Angeles\\
            \textit{e-mail:} kemacdonald@ucla.edu,
            \textit{site}: \href{https://kemacdonald.com}{https://kemacdonald.com}
        \vspace{-1.5ex}
}        
\date{}
\begin{document}
\maketitle
\vspace{-4\baselineskip}
\subsection*{(a) Professional Preparation}
\begin{table}[H]
\centering
\begin{tabular}{llr}
\hline
Institute                                                & Major          & Degree, Year            \\ \hline
Wesleyan University, Middletown, CT                      & Psychology     & Bachelors, 2010       \\
Stanford University, Stanford, CA                        & Developmental Psychology  & Masters, 2016       \\
Stanford University, Stanford, CA                        & Developmental Psychology  & Doctoral, 2018       \\
University of California, Los Angeles, CA                & Communications            & Post-doctoral, 2019
\end{tabular}
\end{table}

\subsection*{(b) Appointments}
\begin{table}[H]
\centering
\begin{tabular}{ll}
2018-present   & Post-doc research scientist, Emergence of Communication Lab, UCLA \\
2016-2017   & Instructor, Developmental Psychology, Stanford University, CA                \\
2013-2018   & Teaching Assistant, Psychology department, Stanford University, CA                   \\
2010-2013   & Research Associate, Language Learning Lab, Stanford University, CA                        
\end{tabular}
\end{table}
\subsection*{(c) Products}

\underline{Related to project}
\nocite{macdonald2018real}
\nocite{macdonald2018noise}
\nocite{macdonald2017information}
\nocite{macdonald2018children}
\nocite{macdonald2019integration}

\bibliographystyle{acm}
\bibliography{macdonald}

\noindent
\underline{Others of significance}

\nociteNew{macdonald2017social}
\nociteNew{macdonald2013my}
\nociteNew{hardwicke2018data} 
\nociteNew{sanchez2019childes}
\nociteNew{yoon2018balancing} 
\nociteNew{barth2014preschoolers}

\bibliographystyleNew{acm}
\bibliographyNew{macdonald}

\subsection*{(d) Synergistic Activities}
\begin{enumerate}

\item Publicly available eye tracking analyses and software development (\href{https://github.com/kemaconald}{https://github.com/kemaconald}). Some examples include:
    \begin{itemize}
        \item \underline{Rtobii}: R Package with utility functions for reading and parsing Tobii eyetracking data [\href{https://github.com/kemacdonald/Rtobii}{link}].
        \item \underline{iChartAnalyzeR}: R-package with utility functions for analyzing eyetracking studies in the style of the Language Learning Lab at Stanford University [\href{https://github.com/kemacdonald/iChartAnalyzeR}{link}].
        \item cluster-based permutation analysis of eye tracking data in R [\href{https://kemacdonald.com/materials/cesana-arlotti_cluster_analysis.nb.html}{link}]
    \end{itemize}

\item Presented research at a 3-day mini-conference in Stockholm (Sweden) on Multimodal Multilingual Outcomes in Deaf and Hard-of-Hearing Children (\href{https://www.ntid.rit.edu/mmoworkshop/}{link}). Work showed how eye movements can be used to measure language abilities in early-identified deaf infants.
\item Organized a symposium at the Society for Research in Child Development on New Approaches to Understanding Human Language: Insights from Neuroimaging and Behavioral Studies of Visual Language Learners. 
\item Presented research on sign language processing at the annual Early Hearing Detection and Intervention Meeting. Discussed with clinicians the evidence that the development of visual language processing follows a similar trajectory as spoken language processing.
\item Publicly available tutorial on Bayesian Linear Mixed Model analyses in the R programming language [\href{https://kemacdonald.com/materials/langcog_rstanarm_tutorial_sleep.nb.html}{link}].
\item Guest lecturer for (1) Linguistics 140 (Sign Language Acquisition), Stanford University and (2) Psych One (Language), Stanford University.
\item Head Teaching Assistant for Stats 60: Intro to Statistical Methods, Stanford University: Developed teaching materials, including an interactive tutorial on exploratory data visualization skills [\href{https://kemacdonald.com/materials/P10_section_week9_viz_students.html}{link}].

\end{enumerate}
\end{document}
